\documentclass{article}
\usepackage[margin=1in]{geometry}
\usepackage{setspace}
\linespread{2}
\begin{document}
\title{Euthanasia: A Most Difficult Decision (Draft)}
\author{Rui Zhang (rz187)}
\maketitle
One of the most contentious topic in modern society and medicine is the ethics of
assisted euthanasia. While many would debate the merits of whether doctors
should even be allowed to euthanize suffering patients, it is almost always readily
agreed upon that, in the event that euthansia is necessary, passive is
always better than active. It is even codified by the American Medical
Association (AMA) and our laws. Many people would also agree with this view, that passive
euthanasia is allowed under situations where prolonging the patient's life
would cause the patient to live out the rest of their lives in extreme
agony. However, philosopher James Rachel would like to argue otherwise.

In his paper, {\em Active and Passive Euthanasia}, he argues that this
commonly held belief is not necessarily morally correct with the following
points:
\begin{enumerate}
    \item By letting a patient die by withholding care, during the time leading up to the
    death the patient is still in immense agony.
    \item The conventional doctrine leads to decisions about life and death being made on irrelevant grounds.
    \item Idly standing by while the patient dies is not morally better than simply taking
    action to end the patient's life.
    \item Therefore, even though it is codified into law and officially endorsed by
    various medical societies, passive euthanasia is at best morally equal to active
    euthanasia and at worst is morally inferior.
\end{enumerate}

First, he argues against the preceived moral superiority of passive euthanasia over
active euthanasia by arguing that during the time between when the physician withholds
care to the point at which point the patient dies, the patient is still in pain and
agony. Rachel argues that once the decision has been made by both the patient and
the physician to end the patient's life for the physician to decide against ending the patient's
life immediately by way of lethal injection is the same as extending the patient's suffering,
which is antithetical to the oaths that the physician takes. He uses the example
of a baby born with Down Syndrome. While many go on to live normal lives with a
bit of medical assistance, some are born with defects that would require surgery in
order to keep them alive. In these cases, instead of letting the child live a life
of agony, they may choose not to operate. He quotes Anthony Shaw, who states that
to do nothing while the infant slowly dies over hours or days is to not only permit
the infant to needlessly suffer but also to cause extreme agony to the surgeons,
nurses, and the parents who can do nothing but watch as their newborn's life fades.
Therefore, it would be better to end the newborn's life immediately than to do nothing
to end the suffering.

Secondly, he points to the fact that, through the traditionally accepted doctrine,
life or death decisions are made based on grounds that have no relevance to the patient's
future or quality of life. He refers to the Down Syndrome example. The decision
whether to save or kill the child is not based on whether or not the child has
Down Syndrome and how it will affect his or her life but entirely based on
the intestinal tract obstruction. If there was no such obstruction, then there
would be no reason to kill the child but now, since an obstruction is present,
there is an opportunity to allow the child to die. By allowing one to decide life
or death based on grounds that are completely irrelevant is, according to Rachel,
a failing of the doctrine and why it should be rejected.

Finally, he questions the idea if killing someone is really morally worse than watching
them die. He presents the example of a person named Smith who will gain a very large
inheritence if his cousin dies. Rachel presents two scenarios. In the first, Smith
kills his cousin in the bathtub and makes it look like an accident. In the second,
the cousin slips and falls in the bathtub while Smith simply stands by and watch. He
is also ready to force the cousin's head back underwater should the cousin manage to
resurface and therefore, the cousin's death would be an accident. Is Smith really
morally better in the latter situation then in the first situation? If it was true
that watching someone die is truly morally better than killing them, then one would
say that Smith watching the cousin die is better than actually killing them.
In both situations, by not saving his cousin, Smith would be, at the very least, be
considered immoral. By the same logic, a doctor who refuses care is not morally
superior to a doctor who administers the lethal injection. In both circumstances, 
the doctor has it in his power to prolong the patient's life but chooses to do the
opposite. The decision that the patient would be better off deceased is the
critical decision a not whether or not any action is taken after the decision.

Some may criticise the above point made by Rachel, saying that there is a
difference between passive and active euthanasia, which is that in passive euthanasia
the doctor does not do anything to bring about the patient's death while in active
euthanasia the doctor does something to bring about the patient's death. However, even in passive euthanasia, the
doctor does do something: deciding to not do anything. That is, the action the doctor
takes is the action of inaction. If by not doing anything a person is not guilty of
a crime then that would mean a doctor who allows a patient to die when the patient
had a very routine and curable illness should not be held accountable for the patient's
death. This would clearly not be the case, as criminal charges would most definitely
be filed against the doctor for his inaction. Therefore, the decision to not do anything
should be judged just as if the doctor had taken some action, according to Rachel.

From the above points, Rachel concludes that while doctors may refrain from performing
active euthanasia because it would make them legally responsible, there is still cause
for concern as the law itself may very well be enforcing a doctrine which may be morally
indefensible. It is for this reason that he opposes the AMA's position on active and
passive euthanasia and that while he believes that doctors should follow the law
medical authorities such as the AMA should not endorse it in their official statements.

Rachel's arguments against the moral superiority of passive euthanasia is very well-articulated.
While superficially, we may look at passive as being better than active euthanasia,
that is most likely only because we have been conditioned that to be the cause of
death of another person is the greatest wrong. Murder, the intentional killing of
an innocent, is the only crime in the United States and many other countries which
carry the death penalty. Therefore, it would only be natural that we look at active
euthanasia, the active action of taking a life, as a sin and an evil. However, when
we consider the cirumstances we can also see why it is justified. Just as Rachel has
articulated, we are not ending a life out of malice, evil intention, or personal gain.
Instead, we are doing so out of mercy. The person is in extreme physical pain that,
even if they receive treatment, will continue to suffer until they eventually die.
Their life would be a slow, painful, torturous death. In these circumstances, the
person may decide that, instead of continuing to live in such agony they believe that
it would be better that they die. In these unfortunate circumstances, if the person
believes that their life is no longer worth living and the physician has judged that
additional treatments would not improve their quality of life, then it should be justified
to release said person from the Hell in which they live.

However, there are a few issues with Rachel's arguments. One of them is the comparison
that he uses to compare active and passive euthanasia, which is that of the cousin.
I do not believe that is an adequate example. In the example scenario, the person
has a reason and motive to facilitate the death of their cousin (i.e. they get a large
sum of money). However, in the case of euthanasia, the doctor, unless they have a
Hannibal Lector-esque personality and derives sick pleasure from death, has nothing
to gain. In fact, it might even be argued that they might have more to lose, such
as having to deliver the bad news to the patient's families and dealing with the knowledge
that they facilitated the death of the patient. Another issue is that Rachel never draws
any line which distinctly states when is it appropriate that a doctor perform
active euthanasia. What if the person is depressed or if they are in a lot of pain
but they still have many years to live? There are certain situations which he did not
account for which passive euthanasia can sidestep. And while the main topic of the
piece is not dealing with when is it appropriate to perform active euthanasia, one
can argue that, because we cannot morally determine this line, we should not be
allowed to perform active euthanasia until we establish this moral boundary. Finally,
there is the argument that a doctor might see active euthanasia as a viable option in
preventing suffering when in reality it should only be considered an option of
last resort. Therefore, while I agree with his premises and conclusion, which is
that active euthanasia is no worse than passive euthanasia, there are a few issues
with the arguments he presented.
\end{document}
