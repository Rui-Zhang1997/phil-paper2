\documentclass{article}
\usepackage[margin=1in]{geometry}
\usepackage{setspace}
\linespread{2}
\begin{document}
\title{Doctor-Assisted Suicide (Draft)}
\author{Rui Zhang (rz187)}
\maketitle
One of the most contemptuous topic in modern society is the ethics of
assisted euthanasia. While many would debate the merits of whether doctors
should even be allowed to euthanize suffering patients, it is almost mostly
agreed upon that, in the event that euthansia is necessary, passive is
always better than active. It is even codified by the American Medical
Association (AMA). Many people would also agree with this view, that passive
euthanasia is allowed under situations where prolonging the patient's life
would cause the patient to live out the rest of their lives in extreme
agony. However, philosopher James Rachel would like to argue otherwise.

In his paper, {\em Active and Passive Euthanasia}, he argues that this
commonly held belief is not necessarily morally correct with the following
points:
\begin{enumerate}
    \item By letting a patient die by withholding care, during the time leading up to the
    death the patient is still in immense agony
    \item The conventional doctrine leads to decisions about life and death being made on irrelevant grounds
    \item Idly standing by while the patient dies is not morally better than simply taking
    action to end the patient's life
\end{enumerate}

First, he argues against the preceived moral superiority of passive euthanasia over
active euthanasia by arguing that during the time between when the physician withholds
care to the point at which point the patient dies, the patient is still in pain and
agony. Rachel argues that, once the decision has been made by both the patient and
the physician to end the patient's life, then to endorse against ending the patient's
life immediately by way of lethal injection is to argue in support of extended suffering,
which is antithetical to the oaths that the physician takes. He uses the example
of a baby born with Down Syndrome. While many go on to live normal lives with a
bit of medical assistance, some are born with defects that would require surgery in
order to keep them alive. In these cases, instead of letting the child live a life
of agony, they may choose not to operate. He quotes Anthony Shaw, who states that
to do nothing while the infant slowly dies over hours or days is to not only permit
the infant to needlessly suffer but also to cause extreme agony to the surgeons,
nurses, and the parents who can do nothing but watch as their newborn's life fades.

Secondly, he points to the fact that, through the traditionally accepted doctrine,
life or death decisions are made based on grounds that have no relevance to the patient's
future or quality of life. He refers to the Down Syndrom example. The decision
whether to save or kill the child is not based on whether or not the child has
Down Syndrome and how it will affect his or her life but entirely based on
the intestinal tract obstruction. If there was no such obstruction, then there
would be no reason to kill the child but now, since an obstruction is present,
there is an opportunity to allow the child to die. By allowing one to decide life
or death based on grounds that are completely irrelevant is, according to Rachel,
a failing of the doctrine and why the doctrine should be rejected.

Finally, he questions the idea if killing someone is really morally worse than watching
them die. He presents the example of a person named Smith who will gain a very large
inheritence if his cousin dies. Rachel presents two scenarios. In the first, Smith
kills his cousin in the bathtub and makes it look like an accident. In the second,
the cousin slips and falls in the bathtub while Smith simply stands by and watch. He
is also ready to force the cousin's head back underwater should the cousin manage to
resurface and therefore, the cousin's death would be an accident. Is Smith really
morally better in the latter situation then in the first situation? If it was true
that watching someone die is truly morally better than killing them, then one would
say that watching gleefully as the cousin dies is better than actually killing them.

Some may criticise the above point made by Rachel, saying that there is, indeed a
difference between passive and active euthanasia, which is that in passive euthanasia
the doctor does not do anything to bring about the patient's death while in active
euthanasia the doctor does something to bring about the patient's death. However, the
doctor does do something: deciding to not do anything. That is, the action the doctor
takes is the action of inaction. If by not doing anything a person is not guilty of
a crime then that would mean a doctor who allows a patient to die when the patient
had a very routine and curable illness should not be held accountable for the patient's
death. This would clearly not be the case, as criminal charges would most definitely
be filed against the doctor for his inaction. Therefore,the decision to not do anything
should be judged just as if the doctor had taken some action, according to Rachel.

From the above points, Rachel concludes that while doctors may refrain from performing
active euthanasia because it would make them legally responsible, there is still cause
for concern as the law itself may very well be enforcing a doctrine which may be morally
indefensible. It is for this reason that he opposes the AMA's position on active and
passive euthanasia and that while he believes that doctors should follow the law
medical authorities such as the AMA should not endorse it in their official statements.

Rachel's points are sound. It is very true, that sometimes there really is no
distinction between doing something or simply watching. Just like how if you see someone
being murdered in broad daylight and you do nothing to assist, whether it be to notify
the authorities or to intervene, would be seen as immoral, the same thing could be argued
for euthanasia. In fact, as Rachel notes towards the end, it may very well be that active
euthanasia is the morally superior choice in this circumstance as passive euthanasia
would do nothing but extend the suffering of the patient and many doctors are not following,
not what they believe is right, but what the law and their medical boards say is right.

However, there is one comparison that Rachel makes which I do not believe properly
illustrates the issue, which is the cousin's example. In the scenario presented, the
person has something to gain from the death of his or her cousin. Therefore, they have
motive in killing their cousin. However, that is not the case in medicine. Unless the
physician has the personality of Hannibal Lector and derives some sick pleasure in
the deaths of their patients, the physician has no gain from allowing their patient to
die. In fact, one could even argue that they have more to lose if they choose to allow
that grim scenario. They would have to face a M\&M board, they might be blamed and 
disparaged by the family of the deceased, and they will most certainly bear the burden
of the person's death, even if they have the comfort that the deceased is better off dead
than alive. It is not in the doctor's playbook to allow or promote death. Their job
is to save the sick. However, I believe if they had used another example, such as a
witness to a crime in progress, the point would more strongly reinforced. Otherwise,
I believe his argument is sound. His point that passive euthanasia is not morally superior
and might very well be morally inferior as it makes the sufferer live for a few more
hours, days, or even weeks or months in pain only to die at the end seems to be cruel.
\end{document}
