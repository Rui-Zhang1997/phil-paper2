\documentclass{article}
\usepackage[margin=1in]{geometry}
\begin{document}
\title{To Die or Not to Die}
\author{Rui Zhang}
\maketitle
Euthanasia, primarily dealing with ending a patient's life in order to end their suffering,
is one of the most contentious topics in modern medicine and society at large. While many
would debate the merits of whether doctors should even be allowed to euthanize suffering
patients, it is almost always readily agreed upon that, in the event that euthanasia is necessary,
passive is always better than active. It is even codified by the American Medical Association (AMA)
and in our laws. Many people would also agree, that passive euthanasia is allowed under circumstances
where prolonging the patient's life would cause the patient to live out their days in extreme
agony. However, philosopher James Rachel would like to argue otherwise. In his paper, Active and
Passive Euthanasia, he argues that the commonly held belief is not necessarily morally superior.

Herein, we will disect his arguments and their validity. Throughout the piece, he has three main points:
\begin{enumerate}
    \item By letting a patient die by withholding care, during the time leading up to death, the patient
    is still in immense agony
    \item Idly standing by while the patient dies is not morally better than simply taking an action to
    end the patient's life
    \item Therefore, active and passive euthanasia are, at best, morally equivalent
\end{enumerate}
First, he argues against the perceived moral superiority of passive euthanasia over active euthanasia by
stating that during the time between when the physician withholds care and when the patient ultimately
dies, the patient is still in a state of great pain and/or agony. Rachel argues that once the decision
has been made by both the patient and the physician that the only way to end the patient's suffering
to end the patient's life, for the physician to then not killing the patient outright would be the same
as extending the patient's suffering without good reason. Something which is antithetical to the oaths
that the physician took. He uses the example of a baby born with Down Syndrome. While many go on to
live normal lives with a bit of medical assistance, some are born with defects that would require surgery
in order to keep on living. In these cases, instead of letting the child live a life of suffering, the
parents may choose not to operate. He quotes Anthony Shaw, who states that to do nothing while the infant
slowly dies over hours or days is to not only permit the infant to needlessly suffer but also
to cause extreme agony to the surgeons, nurses, and the parents who can do nothing but watch as their
newborn's life fades.

In addition, he questions the idea if killing someone is really morally worse than watching them die.
He presents the example of a person named Smith. Smith will gain a very large inheritence if his cousin
dies. Then, Rachel presents two scenarios. In the first, Smith kills his cousin in the bathtub and makes
it look like an accident. In the second, the cousin slips and falls in the bathtub while Smith simply
stands by and watches. He is also ready to force the cousin's head back underwater should the cousin
manage to resurface and therefore, the cousin's death would be an accident. Is Smith really morally
better in the latter situation than in the former? If it is true that watching someone die is truly
morally morally better than killing them, then one would say that Smith watching the cousin die is
better than actually killing them. In both situations, by not saving the cousin, Smith would be,
at the very least, be considered seriously immoral. By the same logic, a doctor who refuses care is
not morally superior to a doctor who administers the lethal injections. In both the presented scenario
and in the issue at hand, both actors, Smith and the physician, has the power to prolong the person's
life (in Smith's case ot save the cousin from drowning and in the physician's case to help keep the
patient alive for a bit longer). However, both of the actors chose not to exercise this power. The
critical decision is, therefore, not whether or not they actively took part in ending a life or not,
but that the decision to end the person's life was made in the first place.

Some may criticise the above point by Rachel, saying that there {\em is} a difference between passive
and active euthanasia, which is that the passive euthanasia the doctor does not do anything to bring
about the patient's death while in active euthanasia the doctor does something to bring about the
patient's death. However, even in passive euthanasia, the doctor does do something: they chose to
not do anything. That is, the action the doctor takes is the action of inaction. If by not doing anything
a person is not guilty of a crime then that would mean a doctor who allows a patient to die that had
a very routine and curable disease should not be held accountable for the patient's death. The physician
did not do anything that would have brought about the patient's death, it was brought about naturally.
However, we all know that this is not the case. Criminal charges against the doctor would likely be filed
for allowing someone to die under his or her care. Therefore, the decision to not do anything should be
judged and weighed just as much as if the doctor had taken some action.

Rachel's arguments against the moral superiority of passive euthanasia is very well-articulated. While
superficially, we may perceive passive euthanasia as being better than active euthanasia, that is most
likely because we have been conditioned to view that to be the cause of death of another person is the
greatest wrong. Murder, the intentional killing of an innocent person, is the only crime in the United
States and elsewhere which carry the death penalty. Therefore, it would only be natural that we look
at active euthanasia, the active action of taking a life, as a sin and a great evil. However, when we
consider the circumstances we can also see why it is justified. Just as Rachel has articulated, we are not
ending a life out of malice, evil intent, or personal gain. Instead, we are doing so out of mercy.
The person is in extreme physical pain or agony that, even if they receive treatment, will continue to suffer
until they eventually die. It would be a slow, painful, torturous death. In these circumstances, the person
may decide that, instead of continuting to live in such agony, they believe that it would be better to die.
In these unfortunate circumstances, if the person believes that their life is no longer worth living
and the physician has judged that additional treatments would not improve the patient's quality of life, then
it should be justified to release the person from the Hell in which they live. Therefore, as I cannot find
any reason to allow any being to continue living in a state of pain that will never be resolved, I agree
with Rachel's sentiment that active euthanasia is at the very least morally equivalent to passive euthanasia.

However, there are a few issues with Rachel's arguments. One of them is the comparison that he uses to 
compare active and passive euthanasia, which is of Smith and the cousin. I do not believe that is an adequate
example. In the example scenario, the person has a reason and motive to facilitate the death of their
cousin (i.e. they get a large sum of money). However, in the case of the doctor, unless they have
a Hannibal Lecter-esque personality and derives sick pleasure from killing, has no motive to kill the patient.
IN fact, it might even be argued that they might have more to lose, such as having to deliver the bad news
to the patient's families and dealing with the knowledge that they facilitated the death of the patient.
Another issue is that Rachel never draws any line which distinctly states when is it appropriate that a doctor
performs active euthanasia. What if the person is depressed or if they are in a lot of pain but still have
many years left to live? There are certain situations which he did not account for which
passive euthanasia can sidestep. And while the main topic of the piece is not concerned with when is it
appropriate to perform active euthanasia, one can argue that, because we cannot make a moral determination
as to when it shoudl be allowed, we should not be allowed to perform active euthanasia until we establish that
moral boundary. Finally, there is the argument that the doctor might see active euthanasia as a viable option in
preventing suffering when in reality it should only be considered an option of last resort. Therefore, while
I agree with his premises and conclusion, which is that active euthanasia is, at the very least morally
equivalent to passive euthanasia, there are some faults with the arguments he presents.
\end{document}
